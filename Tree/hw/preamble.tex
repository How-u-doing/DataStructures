\usepackage[margin=1in]{geometry}
\usepackage[utf8x]{inputenc}
\usepackage{babel} %avoids breaking words of at the end of lines
%\usepackage[numbib]{tocbibind} %Adds "References" to the table of contents
%\usepackage[toc]{appendix}
\usepackage{afterpage}
\usepackage{xcolor}
\usepackage{pdfpages}
\usepackage{wrapfig}
\usepackage{graphicx}
\usepackage{subfig}
\usepackage{caption}

\usepackage{expl3}
\expandafter\def\csname ver@l3regex.sty\endcsname{}
\usepackage{coloremoji}

%New colors defined below for code
\definecolor{codegreen}{rgb}{0,0.6,0}
\definecolor{codegray}{rgb}{0.5,0.5,0.5}
\definecolor{codepurple}{rgb}{0.58,0,0.82}
\definecolor{backcolor}{rgb}{0.95,0.95,0.92}
\definecolor{picturebg}{rgb}{1,0.89,0.80}

%\usepackage{mdframed}
\usepackage[outputdir=./tmp,newfloat]{minted} % need to call 
%<pdflatex -shell-escape>
% minted lineno style
\renewcommand{\theFancyVerbLine}{\sffamily
	\textcolor{codegray}{\scriptsize% \oldstylenums
		{\arabic{FancyVerbLine}}}}
\setminted{
%	mathescape=true,
%	escapeinside=|<>|,
	linenos=true,
	numbersep=1.5mm,
	autogobble,
	breaklines,	breakanywhere,
	fontsize=\footnotesize,
	style=xcode,
	bgcolor=backcolor
}
% Create a new environment for breaking code listings across pages.
\newenvironment{longlisting}{\captionsetup{type=listing}}{}


%% by overleaf %%
\usepackage{listings}
%Code listing style named "mystyle"
\lstdefinestyle{mystyle}{
	backgroundcolor=\color{backcolor},
	commentstyle=\color{codegreen},
	keywordstyle=\color{magenta},
	numberstyle=\tiny\color{codegray},
	stringstyle=\color{codepurple},
	basicstyle=\ttfamily\footnotesize,
	breakatwhitespace=false,         
	breaklines=true,             
	captionpos=b,                    
	keepspaces=true,                 
	numbers=left,                    
	numbersep=5pt,                  
	showspaces=false,                
	showstringspaces=flase,
	showtabs=false,                  
	tabsize=2
}
%"mystyle" code listing set
\lstset{style=mystyle}


\usepackage{fancyhdr}
\pagestyle{fancy}
\fancyhf{} 	% it clears the header and footer of default "plain" page style
\lhead{\leftmark}
\rhead{\thepage}
\chead{\hyperlink{Contents}{Contents}}

\usepackage{hyperref}
\hypersetup{
	colorlinks=true,
	citecolor=red,
	linkcolor=orange,
	urlcolor=magenta
}

\linespread{1.1}
